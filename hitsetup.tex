% !TeX root = ./hitreport-example.tex

% 论文基本信息配置

\hitsetup{
  title  = {哈尔滨工业大学报告 \LaTeX{} \\ 模板使用示例文档},
  expand = {实验一},
  department = {计算学部},
  discipline  = {计算机科学与技术},
  author  = {孙骁},
  student-id = {1180310840},
  supervisor  = {XX},
  lablocation = {格物207},
  term = {2021春季学期},
%  date = {\today},
}

% 载入所需的宏包

% 可以使用 nomencl 生成符号和缩略语说明
% \usepackage{nomencl}
% \makenomenclature

% 表格加脚注
\usepackage{threeparttable}

% 表格中支持跨行
\usepackage{multirow}

% 固定宽度的表格。
% \usepackage{tabularx}

% 跨页表格
\usepackage{longtable}

% 量和单位
\usepackage{siunitx}

% 定理类环境宏包
\usepackage{amsthm}
% 也可以使用 ntheorem
% \usepackage[amsmath,thmmarks,hyperref]{ntheorem}

% 参考文献使用 BibTeX + natbib 宏包
% 顺序编码制
%\usepackage[sort]{natbib}
%\bibliographystyle{thuthesis-numeric}

% 著者-出版年制
% \usepackage{natbib}
% \bibliographystyle{thuthesis-author-year}

% 本科生参考文献的著录格式
% \usepackage[sort]{natbib}
% \bibliographystyle{thuthesis-bachelor}

% 参考文献使用 BibLaTeX 宏包
% \usepackage[backend=biber,style=thuthesis-numeric]{biblatex}
% \usepackage[backend=biber,style=thuthesis-author-year]{biblatex}
% \usepackage[backend=biber,style=apa]{biblatex}
% \usepackage[backend=biber,style=mla-new]{biblatex}
% 声明 BibLaTeX 的数据库
% \addbibresource{ref/refs.bib}

% 定义所有的图片文件在 figures 子目录下
\graphicspath{{figures/}}

% 数学命令
\newcommand\dif{\mathop{}\!\mathrm{d}}  % 微分符号

% hyperref 宏包在最后调用
\usepackage{hyperref}
